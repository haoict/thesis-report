\chapter{Introduction}
\label{ch:introduction}
Nowadays, urban traffic is always one of the most urgent and difficult problem for every city in the world. One of the most vital solution for that problem is encourage passengers using public transportation. However, accquisition of information such as bus route, navigation by bus is somehow difficult for passengers. 
In this thesis, We focus on developing a system that collect and manage buses route information, as well as bring these information to passengers easily by mobile application with friendly user interface, provide fully function such as look up and navigate by bus. 

With the helpful instructions and supports from Dr. Cao Tuan Dung, the thesis was able to ...
The thesis consists of two main parts:

\begin{itemize}
	\item \textbf{Problem statement and solution proposal}: We look for some exist problem about travel by bus. We relize that source of bus information is really limited and the way to approach it. Finally we propose the solution for the problem mentioned above in Section
2.4.	
	\item \textbf{Result achieved}: We analyze and design system, which include database, server side, client side, interact between client and serer, function of management application and interface of client mobile application, etc... After that, we go on with system implementation. Finally is system testing and evalution.
\end{itemize}
%Automated testing has never been an outdated problem in mobile application developing process. There are over 2,500 manufacturer models and over 100 mobile operating system versions \cite{crittercism}. Daily developed mobile applications need to be tested in wide range of phone designs and platforms. These result in enormous amount of testing scenarios and require an effective industrial testing series. Current software-based testing method has shown some advantages but there are still some issues demanded to be dealt with.

%The problem of present testing method is that only software aspect is considered. Tester can run test cases perfectly in software regardless hardware's failure like button or touch screen malfunction. Otherwise, each phone operating system requires different corresponding testing framework. These factors conspire to make the cross-platform mobile application testing very challenging. \nocite{weinman_thesis}

%We propose a new approach that in its design, software and hardware testing are more integrated. Applying image processing technologies, our system can detect content of phone screen and produce actions on it. These actions are performed directly on target phone by delta robot. Robotics testing gives us a less invasive way of mobile testing that does not require special software on target phone. In addition, the testing series can be applied on any mobile phone operating systems without doubt.

% \section{System overview}
% The system overview and design are described below in Figure \ref{fig:sys_overview}.
% In this thesis, we develop a system consist of three components. First, a mobile application as client, which brings buses information to user, provide look up and navigate function, and collect bus route infomation when user traveling by bus. Second, a web service, which provide APIs for client to get and post data. Third, a management applicaiton for manage buses route information, process collected data.

% 	\begin{figure}
% 		\centering
% 		\includegraphics[scale=0.7]{Chapters/Fig/sys_overview.png}
% 		\caption{System overview}
% 		\label{fig:sys_overview}
% 	\end{figure}

% \section{Goals of the thesis}
% In order to achieve desired testing framework, the mission is to complete following tasks:
% 	\begin{itemize}
% 		\item[--] To detect and recognize mobile phone screen's components
% 		\item[--] To generate testing scripts
% 		\item[--] To perform testing process on the robot
% 	\end{itemize}

% \section{Outline of thesis}
% In first three chapters, I will be dealing with technologies and tools supporting testing process which include theoretical background and implementation in the project.

% The next chapter describes practical experiments with the system and our evaluation on the results. Thus can assess the accuracy and reliability of the system.

% In general, whole thesis aims mainly to establish a testing framework on mobile phone using robotics technology. \nocite{radim_thesis}