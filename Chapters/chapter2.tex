\chapter{Mobile phone screen content recognition module}
Mobile phone screen contains much of elements, but can be classified in two group: text and graphical element.
Understanding the meaning of phone screen requires to detect and recognize these elements.
In this chapter, basic objects in screen such as button, text and keyboard are covered.

\section{Technique using for detection}
\subsection{Edge detection}
\subsubsection{Canny Edge Detection}
The main purpose of edge detection is to obtain structure properties of the image with least data extracted for further image processing \cite{canny}. Among some algorithms existing, I choose the one developed by John F. Canny (JFC) in 1986 \cite{jfc_canny}. Known as \gls{ced} or optimal detector, Canny algorithm has three main advantages \cite{code_canny}:
	\begin{itemize}
		\item \textbf{Low error rate}: Meaning a good detection of only existent edges.
		\item \textbf{Good localization}: The distance between edge pixels detected and real edge pixels have to be minimized.
		\item \textbf{Minimal response}: Only one detector response per edge.
	\end{itemize}

\subsubsection{CED algorithm steps}
\begin{itemize}
	\item \textbf{Noise filter}: Using Gaussian filter \cite{gauss_filter}, a sample kernel of \textit{size = 5} might be shown as Equation \ref{eq:gauss}:
	\begin{equation}
		\label{eq:gauss}
		B = \frac{1}{159} .
		\left[ \begin{array}{ccccc}
		2 & 4 & 5 & 4 & 2 \\
		5 & 12 & 15 & 12 & 5 \\
		4 & 9 & 12 & 9 & 4 \\
		2 & 4 & 5 & 4 & 2
		\end{array} \right]
	\end{equation}

	\item \textbf{Find the intensity gradient of the image}: Gradients at each pixel in the smoothed image are determined by applying Sobel-operator \cite{sobel_alg}

	Apply a pair of convolution masks (in \textit{x} and \textit{y} directions:
	\begin{equation}
		\label{eq:gx_gy}
		G_x =
		\left[ \begin{array}{ccc}
		-1 & 0 & 1 \\
		-2 & 0 & 2 \\
		-1 & 0 & 1
		\end{array} \right]
		\qquad
		G_y =
		\left[ \begin{array}{ccc}
		1 & 2 & 1 \\
		0 & 0 & 0 \\
		-1 & -2 & -1
		\end{array} \right]
	\end{equation}

	Find the gradient strength and direction with:
	\begin{equation}
		\label{eq:gradient}
		G = \sqrt{G^2_x + G^2_y}
	\end{equation}
	\begin{equation}
		\label{eq:direction}
		\theta = arctan(\frac{G_y}{G_x})
	\end{equation}
	The direction is rounded to one of four possible angles (namely 0, 45, 90 or 135)

	\item \textbf{Apply Non-maximum suppression}: Removing pixels that are not considered to be part of an edge.
	\item \textbf{Double thresholding}: Potential edges are determined by thresholding.
	\item \textbf{Edge tracking by hysteresis}: Final edges are determined by suppressing all edges that are not connected to a very certain (strong) edge.
\end{itemize}

\subsection{Shape detection}
Detect object by shape (triangle, rectangle, oval)

\subsection{Text detection and extraction}
Morphological operation method
Sobel edge detection algorithm

\subsection{Optical Character Recognition}
\gls{ocr}
Using Tesseract

\section{Recognition result}
